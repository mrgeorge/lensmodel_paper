%-------- DOCUMENT CLASS ------------
\documentclass[12pt]{emulateapj}

%-------- Packages ------------
%\usepackage[iop]{emulateapj}
\usepackage{amsmath,amssymb,natbib,graphicx}
\usepackage[usenames]{color}
\definecolor{Red}{rgb}{1.00,0.00,0.00}
\usepackage[colorlinks,urlcolor=blue,citecolor=black,linkcolor=blue]{hyperref} % for yahapj biblio links
%\pdfoutput=1                  % For the PDF latex compiler

% For US letter size
\special{papersize=8.5in,11in}
\setlength{\pdfpageheight}{\paperheight}
\setlength{\pdfpagewidth}{\paperwidth}


%-------- MACROS ------------

%\setlength{\topmargin}{0.7in} % uncomment when compiling on macbeth

%-------- TITLE COMMENT ------------

 %% You can insert a short comment on the title page using the command below.
%\slugcomment{ }
 
%-------- SHORT TITLE  ---------------------
 
%\shorttitle{}
%\shortauthors{}
 
%-------- BEGIN DOC  ---------------------
 
 \begin{document}
  
%-------- TITLE  ---------------------

 \title{On Measuring Inner Halo Profiles with Weak Lensing Shear and Magnification}

%-------- AUTHORS  ---------------------

 %% Use \author, \affil, and the \and command to format
 %% author and affiliation information.
 %% Note that \email has replaced the old \authoremail command
 %% from AASTeX v4.0. You can use \email to mark an email address
 %% anywhere in the paper, not just in the front matter.
 %% As in the title, use \\ to force line breaks.

\author{Matthew R. George}
\affil{Department of Astronomy, University of California, Berkeley}
\affil{Lawrence Berkeley National Laboratory, 1 Cyclotron Road,
  Berkeley, CA 94720, USA}
\email{mgeorge@astro.berkeley.edu}

%\submitted{Submitted to ApJ}

%-------- ABSTRACT  ---------------------
  
\begin{abstract}
\end{abstract}
 
%-------- KEY WORDS  ---------------------

%\keywords{}

%--------------------------------------------------------------
% INTRODUCTION
%--------------------------------------------------------------

\section{Introduction}

We want to measure the shape of the dark matter profile on small
scales. This will tell us how baryons have affected the dark matter
distribution, the assembly history of the baryons and dark matter as
well as feedback processes, and whether dark matter interacts to form
``cored'' profiles as opposed to the ``cuspy'' profiles seen in cold
dark matter simulations. Measurements of the mass distribution on
scales comparable to the effective radius of a galaxy can also
constrain the mass-to-light ratio of the stellar population which is
uncertain by a factor of $\sim2$ depending on the initial mass
function which is generally assumed.

Various observational probes have been used to constrain the mass
profile of galaxies and dark matter halos, typically with dynamical
measurements within a few kpc of the galaxy center, stong lensing arcs
a bit farther out, and weak lensing on larger scales to measure the
total mass of the halo \citep[e.g.,][]{Sand2004, Koopmans2006,
  Gavazzi2007, Jiang2007, Auger2010, Schulz2010, Newman2013a}.  While
central velocity dispersions or rotation curves can be measured for
large samples of galaxies, detailed kinematic measures on scales
$\gtrsim10$~kpc are difficult and strong lenses exist in only a sparse
sample of the galaxy population. Weak lensing, on the other hand, can
probe the average profile for large samples of galaxies, allowing
studies of population differences and redshift evolution, independent
of their dynamical state. For instance, the assembly histories of disk
and elliptical galaxies may differ and hydrodynamical simulations
predict significant differences in the inner profiles of their dark
matter halos.  The aim of this letter is to investigate how well weak
lensing shear and magnification can constrain the inner mass profile
of dark matter halos and galaxies.

Several authors have studied the complementarity of shear and
magnification, primarily for measuring halo masses
\citep{Bartelmann1996, Bridle1998, Schneider2000, vanWaerbeke2010b,
  Rozo2010, Umetsu2011} or probing the matter distribution on
cosmological scales \citep{vanWaerbeke2010a, Casaponsa2013,
  Duncan2013, Krause2013}. Combining shear and magnification increases
the statistical precision of a lensing experiment and also enables
tests of systematic effects which differ between probes.

Magnification can be measured using sizes \citep{Bartelmann1995},
fluxes \citep{Broadhurst1995}, or by combining the two
\citep{Huff2011, Schmidt2012}. Magnification has recently been
measured with signal-to-noise approaching that from shear for
ensembles of galaxies \citep{Scranton2005, Hildebrandt2009,
  Menard2010, Huff2011, Ford2012, Schmidt2012}.

\section{Lens Modeling}

The distortion of galaxy images due to weak lensing can be described
by a matrix written in terms of the convergence $\kappa$ and shear
components $\gamma_1, \gamma_2$ \citep[e.g.][]{Bartelmann2001}:
\begin{equation}
A = \left( \begin{array}{cc}
1 - \kappa -\gamma_1 & -\gamma_2 \\
-\gamma_2 & 1 - \kappa + \gamma_1 \end{array} \right).
\end{equation}
From galaxy images, one can measure the magnification 
$\mu = (\rm{det}\,A)^{-1} = [(1-\kappa)^2 - |\gamma|^2]^{-1} \approx 1+2\kappa$
and reduced shear $g = \gamma / (1-\kappa) \approx \gamma$, where we have defined the
complex shear $\gamma=\gamma_1 + i\gamma_2$ and approximations are
given to first order in the weak limit $|\gamma|,\kappa \ll 1$.

The convergence and tangential component of shear can be related to the
projected surface mass density $\Sigma$ of the lens via
\begin{equation}
\kappa = \frac{\Sigma}{\Sigma_c}; \,\,\, \gamma_t = \frac{\Delta\Sigma}{\Sigma_c}
\end{equation}
where the critical surface density $\Sigma_{\rm  c}$ is a function of
the angular diameter distances between the observer ($O$), lens ($L$),
and source ($S$),
\begin{equation}
\Sigma_{c}=\frac{c^2}{4\pi G}\frac{D_{OS}}{D_{OL}D_{LS}}.
\end{equation}
A typical experiment averages measurements of $g_t$ or $\kappa$ in
bins of radius $R$ around the lens position to constrain its radial
surface density profile (or stacks many such measurements for an
ensemble of lenses). Magnification directly probes the surface density
at a given position $\Sigma(R)$, whereas shear is sensitive to the
excess surface density interior to the projected radius
$\Delta\Sigma(R) = \overline{\Sigma}(<R) - \overline{\Sigma}(R)$. This
difference in scale-dependence is what we hope to exploit by combining
shear and magnification measurements to constrain inner halo profiles.

We shall assume that $\Sigma$ and $\Delta\Sigma$ can be estimated from
magnification and shear observables in an unbiased manner down to our
minimum radius (typically $40~\rm{kpc}$ which is $\sim20\arcsec$ at
$z_L=0.1$). For discussion of modeling lensing observables into the
nonlinear regime, see \citet{Menard2003, Takada2003,
  Mandelbaum2006}. We will also assume these quantities are
constrained independently; see \citet{Rozo2010} for a treatment of
their covariance.


NFW, gNFW, AC models.

Assumptions: neglect covariance, systematics. Reduced shear? Non-weak
lensing? Relative S/N of shear vs. magnification. No contraction
beyond virial radius, extrapolation with NFW.

Plots of rho, sigma, delta sigma.

% **** FIG *****
\begin{figure*}[htb]
\epsscale{1.3}
\plotone{compareAC}
\caption{Density profiles for different halo models.}
\label{fig:compareAC}
\end{figure*}
% **** FIG *****


\section{Forecasts}

Results. Choice of fiducial system(s) - galaxy vs cluster, concentration, mention
miscentering.

Priors

Plots of constraints.

\section{Conclusions}


 %-------------- ACKNOWLEDGMENTS --------------------

\acknowledgments 

\mbox{~} % needed to prevent chopping off the last line above
%-------------- BIBLIO --------------------------------

%\pagebreak

%\bibliographystyle{yahapj}
\bibliographystyle{apj}
\bibliography{lensmodel}

 %----------------------------------------------------

\end{document}
