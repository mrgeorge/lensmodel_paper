%-------- DOCUMENT CLASS ------------
\documentclass[12pt]{emulateapj}

%-------- Packages ------------
%\usepackage[iop]{emulateapj}
\usepackage{amsmath,amssymb,natbib,graphicx}
\usepackage[usenames]{color}
\definecolor{Red}{rgb}{1.00,0.00,0.00}
\usepackage[colorlinks,urlcolor=blue,citecolor=black,linkcolor=blue]{hyperref} % for yahapj biblio links
%\pdfoutput=1                  % For the PDF latex compiler

% For US letter size
\special{papersize=8.5in,11in}
\setlength{\pdfpageheight}{\paperheight}
\setlength{\pdfpagewidth}{\paperwidth}


%-------- MACROS ------------

%\setlength{\topmargin}{0.7in} % uncomment when compiling on macbeth

%-------- TITLE COMMENT ------------

 %% You can insert a short comment on the title page using the command below.
%\slugcomment{ }
 
%-------- SHORT TITLE  ---------------------
 
%\shorttitle{}
%\shortauthors{}
 
%-------- BEGIN DOC  ---------------------
 
 \begin{document}
  
%-------- TITLE  ---------------------

 \title{On Measuring Inner Halo Profiles with Weak Lensing Shear and Magnification}

%-------- AUTHORS  ---------------------

 %% Use \author, \affil, and the \and command to format
 %% author and affiliation information.
 %% Note that \email has replaced the old \authoremail command
 %% from AASTeX v4.0. You can use \email to mark an email address
 %% anywhere in the paper, not just in the front matter.
 %% As in the title, use \\ to force line breaks.

\author{Matthew R. George}
\affil{Department of Astronomy, University of California, Berkeley}
\affil{Lawrence Berkeley National Laboratory, 1 Cyclotron Road,
  Berkeley, CA 94720, USA}
\email{mgeorge@astro.berkeley.edu}

%\submitted{Submitted to ApJ}

%-------- ABSTRACT  ---------------------
  
\begin{abstract}
\end{abstract}
 
%-------- KEY WORDS  ---------------------

%\keywords{}

%--------------------------------------------------------------
% INTRODUCTION
%--------------------------------------------------------------

\section{Introduction}

We want to measure the shape of the dark matter profile on small
scales. This will tell us how baryons have affected the dark matter
distribution, the assembly history of the baryons and dark matter as
well as feedback processes, and whether dark matter interacts to form
``cored'' profiles as opposed to the ``cuspy'' profiles seen in cold
dark matter simulations. Measurements of the mass distribution on
scales comparable to the effective radius of a galaxy can also
constrain the mass-to-light ratio of the stellar population which is
uncertain by a factor of $\sim2$ depending on the initial mass
function which is generally assumed.

Various observational probes have been used to constrain the mass
profile of galaxies and dark matter halos, typically with dynamical
measurements within a few kpc of the galaxy center, stong lensing arcs
a bit farther out, and weak lensing on larger scales to measure the
total mass of the halo \citep[e.g.,][]{Sand2004, Koopmans2006,
  Gavazzi2007, Jiang2007, Auger2010, Schulz2010, Newman2013a}.  While
central velocity dispersions or rotation curves can be measured for
large samples of galaxies, detailed kinematic measures on scales
$\gtrsim10$~kpc are difficult and strong lenses exist in only a sparse
sample of the galaxy population. Weak lensing, on the other hand, can
probe the average profile for large samples of galaxies, allowing
studies of population differences and redshift evolution, independent
of their dynamical state. For instance, the assembly histories of disk
and elliptical galaxies may differ and hydrodynamical simulations
predict significant differences in the inner profiles of their dark
matter halos. Additionally, weak lensing shear and magnification have
different scale-dependence on the surface mass density profile
providing added leverage in constraining its shape. The aim of this
letter is to investigate how well weak lensing shear and magnification
can constrain the inner mass profile of dark matter halos and
galaxies.

Several authors have studied the complementarity of shear and
magnification, primarily for measuring halo masses
\citep{Bartelmann1996, Bridle1998, Schneider2000, vanWaerbeke2010b,
  Rozo2010, Umetsu2011} or probing the matter distribution on
cosmological scales \citep{vanWaerbeke2010a, Casaponsa2013,
  Duncan2013, Krause2013}. Combining shear and magnification increases
the statistical precision of a lensing experiment and also enables
tests of systematic effects which differ between probes.

While weak lensing experiments traditionally measure galaxy
ellipticities to infer the shear signal, magnification can be measured
using sizes \citep{Bartelmann1995}, fluxes \citep{Broadhurst1995}, or
by combining the two \citep{Huff2011, Schmidt2012}. Magnification has
recently been measured with signal-to-noise approaching that from
shear for ensembles of galaxies \citep{Scranton2005, Hildebrandt2009,
  Menard2010, Huff2011, Ford2012, Schmidt2012}.

THIS PAPER

We define halos within a virial overdensity of $200$ times the
critical density of the Universe and use physical distances with
$h=0.7, \Omega_m=0.3, \Omega_\Lambda=0.7$.

\section{Lens Modeling}

The distortion of galaxy images due to weak lensing can be described
by a matrix written in terms of the convergence $\kappa$ and shear
components $\gamma_1, \gamma_2$ \citep[e.g.][]{Bartelmann2001}:
\begin{equation}
A = \left( \begin{array}{cc}
1 - \kappa -\gamma_1 & -\gamma_2 \\
-\gamma_2 & 1 - \kappa + \gamma_1 \end{array} \right).
\end{equation}
From galaxy images, one can measure the magnification 
$\mu = (\rm{det}\,A)^{-1} = [(1-\kappa)^2 - |\gamma|^2]^{-1} \approx 1+2\kappa$
and reduced shear $g = \gamma / (1-\kappa) \approx \gamma$, where we have defined the
complex shear $\gamma=\gamma_1 + i\gamma_2$ and approximations are
given to first order in the weak limit $|\gamma|,\kappa \ll 1$.

The convergence and tangential component of shear can be related to the
projected surface mass density $\Sigma$ of the lens via
\begin{equation}
\kappa = \frac{\Sigma}{\Sigma_c}; \,\,\, \gamma_t = \frac{\Delta\Sigma}{\Sigma_c}
\end{equation}
where the critical surface density $\Sigma_{\rm  c}$ is a function of
the angular diameter distances between the observer ($O$), lens ($L$),
and source ($S$),
\begin{equation}
\Sigma_{c}=\frac{c^2}{4\pi G}\frac{D_{OS}}{D_{OL}D_{LS}}.
\end{equation}
A typical galaxy lensing experiment averages measurements of $g_t$ or $\kappa$ in
bins of radius $R$ around the lens position to constrain its radial
surface density profile (or stacks many such measurements for an
ensemble of lenses). Magnification directly probes the surface density
at a given position $\Sigma(R)$, whereas shear is sensitive to the
excess surface density interior to the projected radius
$\Delta\Sigma(R) = \overline{\Sigma}(<R) - \overline{\Sigma}(R)$. This
difference in scale-dependence is what we hope to exploit by combining
shear and magnification measurements to constrain inner halo profiles.

We shall assume that $\Sigma$ and $\Delta\Sigma$ can be estimated from
magnification and shear observables in an unbiased manner down to our
minimum radius (typically $40~\rm{kpc}$ which is $\sim20\arcsec$ at
$z_L=0.1$). For discussion of modeling lensing observables into the
nonlinear regime, see \citet{Menard2003, Takada2003,
  Mandelbaum2006}. We will also assume these quantities are
constrained independently; see \citet{Rozo2010} for a treatment of
their covariance.

We consider parametric models for the three-dimensional density
profile $\rho(r)$ of the lens galaxy and dark matter halo, which we
then project into two dimensions for comparison with the lensing
observables. For the stellar component, we assume a
\citet{Hernquist1990} profile of the form
\begin{equation}
\rho_{\star}(r) = \frac{M_{\star}}{2\pi} \frac{a}{r(r+r_{\star})^3}
\label{eq:hernquist}
\end{equation}
parametrized by the total stellar mass $M_{\star}$ and scale radius
$r_{\star}$, which is a good description of elliptical galaxies. If a
constant stellar mass-to-light ratio is assumed, the 
projected scale radius $R_{\rm deV}$ measured from fitting a
de~Vaucouleurs model to the surface brightness profile of the galaxy
can be converted to the three-dimension Hernquist radius using $R_{\rm
  deV}=1.8153\,r_{\star}$. Stellar masses are typically
estimated by modeling the spectral energy distribution of a galaxy,
but assumptions in this process lead to significant systematic
uncertainties dominated by our ignorance of the stellar initial mass
function (IMF). We will treat $M_{\star}$ as a free parameter to
see how well lensing can constrain these uncertainties.

Our baseline model for the dark matter halo is the Navarro-Frenk-White
\citep[NFW, ][]{Navarro1996} profile with parameters for the halo
mass~$M_{h}$ and concentration~$c$. \citet{Wright2000} give
projections of this profile to compute $\Sigma_{\rm NFW}$ and
$\Delta\Sigma_{\rm NFW}$.

The NFW model was introduced to describe the form of halos in dark
matter simulations, but it is known that the baryonic processes
including cooling and feedback can modify the shape of the dark matter
profile \citep[e.g.,][]{Blumenthal1986, Gao2004, Gnedin2004,
  Johansson2009, Abadi2010, Gnedin2011}. Because we lack a good
physical description of these processes, we consider a variable amount
of halo contraction or expansion using the model of
\citet{Dutton2007}. This model assumes the quantity $rM(r)^\nu$ is
conserved during the formation of a galaxy, with both baryons and dark
matter initially distributed following an NFW profile, and the baryons
eventually collapsing into a final distribution which we take to be
$\rho_{\star}(r)$ from Equation~\ref{eq:hernquist}. The parameter
$\nu$ controls the amount of contraction, with $\nu=1$ recovering the
adiabatic contraction (AC) of \citet{Blumenthal1986}, $\nu=0$ giving an
uncontracted NFW profile, and $\nu<0$ for expansion of the dark matter
on small scales. Hydrodynamical simulations predict a range of values
$-0.2\lesssim \nu \lesssim 0.8$ (CHECK?) depending on the details of
the cooling and feedback models as well as the assembly history of the
galaxy.

We note that these modified halo models are typically implemented
assuming that the mass within the virial radius is conserved, i.e.,
that there is no contraction at or beyond the virial radius. While
contraction models produce an excess mass density at $r \lesssim
r_{\star}$, this boundary condition leads to a mass \textit{deficit} on
intermediate scales between $\sim r_{\star}$ and the virial radius, and vice
versa for expansion models. We demonstrate this effect in
Figure~\ref{fig:compareAC} showing $\rho, \Sigma,$ and $\Delta\Sigma$
halo profiles with model parameters from the faint
early-type sample in Table 1 of \citet{Schulz2010}. We compare profiles that have
undergone expansion ($\nu=-0.2$) or adiabatic contraction to NFW
profiles with the same initial halo parameters. To compute the
projected profiles we extrapolate $\rho(r)$ from $1-10$ virial radii
using the initial NFW profile. The AC model deviates
by $\sim25\%$ from the corresponding NFW model at scales of tens of
kiloparsecs. We also see that the shear and magnification observables have
different scale-dependence demonstrating their complementarity.

Figure~\ref{fig:compareAC} includes a stellar profile for comparison
to the halo models. We see that the stars fall off much more rapidly
than the dark matter and that the transition between excess and
deficit regions for the AC model relative to NFW occurs at roughly the scale
where stars and dark matter contribute equally to the total
profile. In practice, lensing can only measure the sum of the stellar
and dark matter components but we separate these components in the
figure to isolate the contraction/expansion effects. 

The contraction model depends on the ratios $M_{\star}/M_h$ and
$r_{\star}/r_h$, with greater contraction for galaxies that are more
massive and compact relative to their halos. For a galaxy with
$M_{\star}=10^{11}~M_{\odot}, R_{\rm deV}=10~{\rm kpc}$ in a
cluster-scale halo with $M_{h}=10^{15}~M_{\odot}, c=5$, the AC and
expansion models deviate by less than $5\%$ from the NFW profile at
$R>10~{\rm kpc}$. The value of $M_{\star}/M_{h}$ peaks near
$M_{\star}\approx10^{10.5}, M_h\approx10^{12}$
\citep[e.g.][]{Conroy2009, Behroozi2010, Leauthaud2012}, which
motivates targeting compact galaxies in this mass range, rather than
massive clusters or low-mass dwarfs, to measure baryonic contraction
effects. Additionally, weak lensing measurements are often limited on
small angular scales due to systematic effects, so a low-redshift lens
sample allows one to probe smaller physical scales where contraction
effects are greatest.


% **** FIG *****
\begin{figure*}[htb]
\epsscale{1.3}
\plotone{compareAC}
\caption{Density profiles for different halo models. [Add stars for
  comparison, Use correct 2d-3d Rdev.]}
\label{fig:compareAC}
\end{figure*}
% **** FIG *****


\section{Forecasts}

Results. Choice of fiducial system(s) - galaxy vs cluster, concentration, mention
miscentering.

Relative S/N of shear vs magnification.

Priors

Plots of constraints.

\section{Conclusions}


 %-------------- ACKNOWLEDGMENTS --------------------

\acknowledgments 

\mbox{~} % needed to prevent chopping off the last line above
%-------------- BIBLIO --------------------------------

%\pagebreak

%\bibliographystyle{yahapj}
\bibliographystyle{apj}
\bibliography{lensmodel}

 %----------------------------------------------------

\end{document}
