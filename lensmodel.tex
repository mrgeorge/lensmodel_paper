%-------- DOCUMENT CLASS ------------
\documentclass[12pt]{emulateapj}

%-------- Packages ------------
%\usepackage[iop]{emulateapj}
\usepackage{amsmath,amssymb,natbib,graphicx,bm}
\usepackage[usenames]{color}
\definecolor{Red}{rgb}{1.00,0.00,0.00}
\usepackage[colorlinks,urlcolor=blue,citecolor=black,linkcolor=blue]{hyperref} % for yahapj biblio links
%\pdfoutput=1                  % For the PDF latex compiler

% For US letter size
\special{papersize=8.5in,11in}
\setlength{\pdfpageheight}{\paperheight}
\setlength{\pdfpagewidth}{\paperwidth}


%-------- MACROS ------------

%\setlength{\topmargin}{0.7in} % uncomment when compiling on macbeth

%-------- TITLE COMMENT ------------

 %% You can insert a short comment on the title page using the command below.
%\slugcomment{ }
 
%-------- SHORT TITLE  ---------------------
 
%\shorttitle{}
%\shortauthors{}
 
%-------- BEGIN DOC  ---------------------
 
 \begin{document}
  
%-------- TITLE  ---------------------

 \title{On Measuring Inner Halo Profiles with Weak Lensing Shear and Magnification}

%-------- AUTHORS  ---------------------

 %% Use \author, \affil, and the \and command to format
 %% author and affiliation information.
 %% Note that \email has replaced the old \authoremail command
 %% from AASTeX v4.0. You can use \email to mark an email address
 %% anywhere in the paper, not just in the front matter.
 %% As in the title, use \\ to force line breaks.

\author{Matthew R. George}
\affil{Department of Astronomy, University of California, Berkeley}
\affil{Lawrence Berkeley National Laboratory, 1 Cyclotron Road,
  Berkeley, CA 94720, USA}
\email{mgeorge@astro.berkeley.edu}

%\submitted{Submitted to ApJ}

%-------- ABSTRACT  ---------------------
  
\begin{abstract}
\end{abstract}
 
%-------- KEY WORDS  ---------------------

%\keywords{}

%--------------------------------------------------------------
% INTRODUCTION
%--------------------------------------------------------------

\section{Introduction}

We want to measure the shape of the dark matter profile on small
scales. This will tell us how baryons have affected the dark matter
distribution, the assembly history of the baryons and dark matter as
well as feedback processes, and whether dark matter interacts to form
``cored'' profiles as opposed to the ``cuspy'' profiles seen in cold
dark matter simulations. Measurements of the mass distribution on
scales comparable to the effective radius of a galaxy can also
constrain the mass-to-light ratio of the stellar population which is
uncertain by a factor of $\sim2$ depending on the initial mass
function which is generally assumed.

Various observational probes have been used to constrain the mass
profile of galaxies and dark matter halos, typically with dynamical
measurements within a few kpc of the galaxy center, stong lensing arcs
a bit farther out, and weak lensing on larger scales to measure the
total mass of the halo \citep[e.g.,][]{Sand2004, Koopmans2006,
  Gavazzi2007, Jiang2007, Auger2010, Schulz2010, Newman2013a}.  While
central velocity dispersions or rotation curves can be measured for
large samples of galaxies, detailed kinematic measures on scales
$\gtrsim10\,{\rm kpc}$ are difficult and strong lenses exist in only a sparse
sample of the galaxy population. Weak lensing, on the other hand, can
probe the average profile for large samples of galaxies, allowing
studies of population differences and redshift evolution, independent
of their dynamical state. For instance, the assembly histories of disk
and elliptical galaxies may differ and hydrodynamical simulations
predict significant differences in the inner profiles of their dark
matter halos. Additionally, weak lensing shear and magnification have
different scale-dependence on the surface mass density profile
providing added leverage in constraining its shape. The aim of this
letter is to investigate how well weak lensing shear and magnification
can constrain the inner mass profile of dark matter halos and
galaxies.

Several authors have studied the complementarity of shear and
magnification, primarily for measuring halo masses
\citep{Bartelmann1996, Bridle1998, Schneider2000, vanWaerbeke2010b,
  Rozo2010, Umetsu2011} or probing the matter distribution on
cosmological scales \citep{vanWaerbeke2010a, Casaponsa2013,
  Duncan2013, Krause2013}. Combining shear and magnification increases
the statistical precision of a lensing experiment and also enables
tests of systematic effects which differ between probes.

While weak lensing experiments traditionally measure galaxy
ellipticities to infer the shear signal, magnification can be measured
using sizes \citep{Bartelmann1995}, fluxes \citep{Broadhurst1995}, or
by combining the two \citep{Huff2011, Schmidt2012}. Magnification has
recently been measured with signal-to-noise approaching that from
shear for ensembles of galaxies \citep{Scranton2005, Hildebrandt2009,
  Menard2010, Huff2011, Ford2012, Schmidt2012}.

THIS PAPER

We define halos within a virial overdensity of $200$ times the
critical density of the Universe and use physical distances with
$h=0.7, \Omega_m=0.3, \Omega_\Lambda=0.7$. Logarithmic quantities
denoted with $\log$ and $\ln$ implicitly use base ten and $e$ respectively.

\section{Lens Modeling}
\label{sec:model}

The distortion of galaxy images due to weak lensing can be described
by a matrix written in terms of the convergence $\kappa$ and shear
components $\gamma_1, \gamma_2$ \citep[e.g.][]{Bartelmann2001}:
\begin{equation}
A = \left( \begin{array}{cc}
1 - \kappa -\gamma_1 & -\gamma_2 \\
-\gamma_2 & 1 - \kappa + \gamma_1 \end{array} \right).
\end{equation}
From galaxy images, one can measure the magnification 
$\mu = (\rm{det}\,A)^{-1} = [(1-\kappa)^2 - |\gamma|^2]^{-1} \approx 1+2\kappa$
and reduced shear $g = \gamma / (1-\kappa) \approx \gamma$, where we have defined the
complex shear $\gamma=\gamma_1 + i\gamma_2$ and approximations are
given to first order in the weak limit $|\gamma|,\kappa \ll 1$.

The convergence and tangential component of shear can be related to the
projected surface mass density $\Sigma$ of the lens via
\begin{equation}
\kappa = \frac{\Sigma}{\Sigma_c}; \,\,\, \gamma_t = \frac{\Delta\Sigma}{\Sigma_c}
\end{equation}
where the critical surface density $\Sigma_{\rm  c}$ is a function of
the angular diameter distances between the observer ($o$), lens ($l$),
and source ($s$),
\begin{equation}
\Sigma_{c}=\frac{c^2}{4\pi G}\frac{D_{os}}{D_{ol}D_{ls}}.
\end{equation}
A typical galaxy lensing experiment averages measurements of $g_t$ or $\mu$ in
bins of radius $R$ around the lens position to constrain its radial
surface density profile (or stacks many such measurements for an
ensemble of lenses). Magnification directly probes the surface density
at a given position $\Sigma(R)$, whereas shear is sensitive to the
excess surface density interior to the projected radius
$\Delta\Sigma(R) = \overline{\Sigma}(<R) - \overline{\Sigma}(R)$. This
difference in scale-dependence is what we hope to exploit by combining
shear and magnification measurements to constrain inner halo profiles.

We consider parametric models for the three-dimensional density
profile $\rho(r)$ of the lens galaxy and dark matter halo, which are
then projected into two dimensions for comparison with the lensing
observables. The stellar component is modeled as a
\citet{Hernquist1990} profile of the form
\begin{equation}
\rho_{\star}(r) = \frac{M_{\star}}{2\pi} \frac{a}{r(r+r_{\star})^3}
\label{eq:hernquist}
\end{equation}
parametrized by the total stellar mass $M_{\star}$ and scale radius
$r_{\star}$, which is a good description of elliptical galaxies. If a
constant stellar mass-to-light ratio is assumed, the 
projected scale radius $R_{\rm deV}$ measured from fitting a
de~Vaucouleurs model to the surface brightness profile of the galaxy
can be converted to the three-dimensional Hernquist radius using $R_{\rm
  deV}=1.8153\,r_{\star}$. Stellar masses are typically
estimated by modeling the spectral energy distribution of a galaxy,
but assumptions in this process lead to significant systematic
uncertainties dominated by our ignorance of the stellar initial mass
function (IMF). We will treat $M_{\star}$ as a free parameter to
determine how well lensing can constrain these uncertainties.

Our baseline model for the dark matter halo is the Navarro-Frenk-White
\citep[NFW, ][]{Navarro1996} profile with parameters for the halo
mass~$M_{h}$ and concentration~$c$. \citet{Wright2000} give
projections of this profile to compute $\Sigma_{\rm NFW}$ and
$\Delta\Sigma_{\rm NFW}$.

The NFW model was introduced to describe the form of halos in dark
matter simulations, but it is known that baryonic processes
including cooling and feedback can modify the shape of the dark matter
profile \citep[e.g.,][]{Blumenthal1986, Gao2004, Gnedin2004,
  Johansson2009, Abadi2010, Gnedin2011}. Because we lack a good
physical description of these processes, we consider a variable amount
of halo contraction or expansion using the model of
\citet{Dutton2007}. This model assumes the quantity $rM(r)^\nu$ is
conserved during the formation of a galaxy, with both baryons and dark
matter initially distributed following an NFW profile, and the baryons
eventually collapsing into a final distribution which we take to be
$\rho_{\star}(r)$ from Equation~\ref{eq:hernquist}. The parameter
$\nu$ controls the amount of contraction, with $\nu=1$ recovering the
adiabatic contraction (AC) of \citet{Blumenthal1986}, $\nu=0$ giving an
uncontracted NFW profile, and $\nu<0$ for expansion of the dark matter
on small scales. Hydrodynamical simulations predict a range of values
$-0.2\lesssim \nu \lesssim 0.8$ (CHECK?) depending on the details of
the cooling and feedback models as well as the assembly history of the
galaxy.

Models of halo contraction are typically implemented
assuming that the mass within the virial radius is conserved, i.e.,
that there is no contraction at or beyond the virial radius. While
contraction models produce an excess mass density at $r \lesssim
r_{\star}$, this boundary condition leads to a mass \textit{deficit} on
intermediate scales between $\sim r_{\star}$ and the virial radius, and vice
versa for expansion models. We demonstrate this effect in
Figure~\ref{fig:compareAC} showing $\rho, \Sigma,$ and $\Delta\Sigma$
halo profiles with model parameters given in
Table~\ref{tab:model}. The fiducial NFW profile is compared to models
with the same initial halo 
parameters that have undergone expansion ($\nu=-0.2$) or adiabatic
contraction. To compute the projected profiles we extrapolate
$\rho(r)$ beyond the virial radius using the initial NFW profile. The AC model deviates
by $\sim25\%$ from the corresponding NFW model at scales of tens of
kiloparsecs. We also see that the shear and magnification observables have
different scale-dependence demonstrating their complementarity.

% **** FIG *****
\begin{figure*}[htb]
\epsscale{1.3}
\plotone{compareAC}
\caption{Density ($\rho(r)$; left), magnification ($\Sigma(R)$;
  center), and shear ($\Delta\Sigma(R)$; right) profiles for different halo models showing the
  effects of baryonic contraction or expansion relative to NFW. A
  Hernquist profile for the stellar component is included for
  comparison. Stellar and NFW model parameters are from Table~\ref{tab:model}. The AC model uses
  $\nu=1$ while the Expansion model has $\nu=-0.2$. The boundary
  condition specifying no contraction at the virial radius leads to a
  deficit (excess) in the AC (Expansion) profiles relative to NFW at
  $r_{\star}\lesssim r\lesssim r_{\rm vir}$.}
\label{fig:compareAC}
\end{figure*}
% **** FIG *****

Figure~\ref{fig:compareAC} includes a stellar profile for comparison
to the halo models. The stars fall off much more rapidly
than the dark matter and the transition between the excess and
deficit regions for the AC model relative to NFW occurs at roughly the scale
where stars and dark matter contribute equally to the total
profile. In practice, lensing can only measure the sum of the stellar
and dark matter components but we separate these components in the
figure to isolate the contraction/expansion effects. 

The contraction model depends on the ratios $M_{\star}/M_h$ and
$r_{\star}/r_h$, with greater contraction for galaxies that are more
massive and compact relative to their halos. For a galaxy with
$M_{\star}=10^{11}\,M_{\odot}, R_{\rm deV}=10\,{\rm kpc}\,
(r_\star=5.5\,{\rm kpc})$ in a
cluster-scale halo with $M_{h}=10^{15}\,M_{\odot}, c=5$, the AC and
expansion models deviate by less than $3\%$ from the NFW profile at
$R>10\,{\rm kpc}$. The value of $M_{\star}/M_{h}$ peaks near
$M_{\star}\approx10^{10.5}, M_h\approx10^{12}$
\citep[e.g.][]{Conroy2009, Behroozi2010, Leauthaud2012} which
motivates targeting compact galaxies in this mass range to measure
baryonic contraction effects, rather than massive clusters or low-mass
dwarfs. Additionally, weak lensing measurements are often limited on 
small angular scales due to systematic effects, so a low-redshift lens
sample allows one to probe smaller physical scales where contraction
effects are greatest.


\section{Forecasts}

To understand how well model halo profiles can be constrained, we need
to estimate the signal-to-noise (S/N) of the lensing measurement. We
choose two fiducial galaxy models and two sets of survey parameters
given in Tables~\ref{tab:model} and~\ref{tab:survey}.  The fiducial
models are constructed as in Section~\ref{sec:model}, with the total
signal composed of the sum of an NFW halo and a Hernquist profile for
the stellar component. For each of the the lens samples which have
mean stellar masses of $\log(M_{\star}/M_{\odot})=10.5$ (``Galaxies'')
and $11.5$ (``Clusters''), the halo mass is taken from the mean
$M_h(M_\star)$ relation derived from abundance matching
\citep{Behroozi2010} and the concentration is set to the mean $c(M_h)$
relation from recent dark matter simulations \citep{Klypin2011}. The
number densities for the lens samples are estimated from a fit to an
empirical stellar mass function \citep{Li2009} assuming galaxies are
selected in bins of width $\Delta\log(M_{\star})=0.5\,{\rm dex}$ and
$\Delta z_l=0.1$. Survey parameters reflect the existing shape catalog
in SDSS \citep{Reyes2012} and anticipated shear data from LSST
\citep{Chang2013}.

\begin{deluxetable*}{llccl}
\tablecaption{Model Parameters}
\tablehead{ \colhead{Name} & \colhead{Description} & \colhead{Galaxies} & \colhead{Clusters} & \colhead{Prior}}
\startdata
$\log(M_{\star})$ & Stellar mass $(M_{\odot})$ & 10.5 & 11.5 & Flat[fid.$\pm0.5$] \\
$\log(r_{\star})$ & Hernquist radius $(\rm{kpc})$ & 0.5 & 0.5 & Fixed \\
$\log(M_{h})$ & Halo mass $(M_{\odot}, \Delta=200\rho_{c})$ & 11.968 & 14.745 & Flat[fid.$\pm0.5$] \\
$c$ & Halo concentration & 7.44 & 4.60 & Flat[2,10]; Fixed \\
$\nu$ & Contraction level & 0 & 0 & Flat[-0.2, 1.0] \\
$n_l$ & Lens density $({\rm deg}^{-2})$ & 39.5 & 0.0947 & -
\enddata
\label{tab:model}
\end{deluxetable*}

\begin{deluxetable}{llcc}
\tablecaption{Survey Parameters}
\tablehead{ \colhead{Name} & \colhead{Description} & \colhead{SDSS} & \colhead{LSST}}
\startdata
$A_S$ & Survey area $({\rm deg}^2)$ & 9243 & 18000 \\
$n_s$ & Source density $({\rm arcmin}^{-2})$ & 1.2 & 37 \\
$z_l$ & Lens redshift & 0.1 & 0.1 \\
$z_s$ & Source redshift & 0.39 & 0.82
\enddata
\label{tab:survey}
\end{deluxetable}

The measurement uncertainty for shear is assumed to be dominated by
intrinsic shape noise and is estimated as
\begin{equation}
\sigma_{\Delta\Sigma}=\frac{\sigma_{\gamma}\Sigma_c}{\sqrt{n_l A_S n_s
  \pi(R_2^2-R_1^2)}}
\end{equation}
where the denominator is the square root of the number of source-lens
pairs in the survey and $R_1$ and $R_2$ are the edges of the radial
bins. For the shape noise, we take
$\sigma_{\gamma}=0.25$ as the intrinsic scatter in a single component
of $\gamma$ due to the variety of galaxy ellipticities, noting that this
value falls between measured values in SDSS \citep{Hirata2004} and
COSMOS \citep{Leauthaud2007}. The variety of approaches to
magnification measurements mentioned in the introduction have
differences in their source selection and in the intrinsic scatter for
the observable. While these properties differ from shear experiments
in general, the measurement uncertainty typically 
has the same scaling with $\Sigma_c, n_l, A_S,$ and $R$. We thus
parametrize the measurement uncertainty for magnification with 
\begin{equation}
\Gamma = \frac{\sigma_{\Delta\Sigma}}{\sigma_{\Sigma}} =
\frac{\sigma_\gamma}{\sigma_\kappa}\sqrt{\frac{n_{s, \rm{mag}}}{n_{s, \rm{shear}}}},
\end{equation}
and use values of $\Gamma=0.5, 1$ to represent current and optimistic
magnification measurements. The fiducial models are shown in
Figure~\ref{fig:fiducial} with measurement errors for each survey
plotted in the range $20-2000\,{\rm kpc}$ with bins of width
$\Delta\log(R)=0.15$.

% **** FIG *****
\begin{figure*}[htb]
\epsscale{1.3}
\plotone{fiducial}
\caption{Magnification ($\Sigma(R)$; left) and shear
  ($\Delta\Sigma(R)$; right) profiles for fiducial galaxy and cluster models with
  parameters given in Table~\ref{tab:model} and error bars predicted
  from the survey parameters in Table~\ref{tab:survey}. Errors for
  both surveys on the cluster sample and for LSST on the galaxy sample
are similar to the line widths.}
\label{fig:fiducial}
\end{figure*}
% **** FIG *****


We make a number of idealizing assumptions since our goal is to
illustrate the potential of such measurements and to motivate
development of methods to deal with small-scale systematics. For
simplicity, we take a single value for the stellar and halo masses as well as
the lens and source redshifts, rather than accounting for the full
distribution of each parameter. We assume that $\Sigma$ and
$\Delta\Sigma$ can be estimated from magnification and shear
observables in an unbiased manner down to our minimum radius ($R_{\rm
  min} = 20$ or $40\,\rm{kpc}$ corresponding to angular scales of
$\sim10\arcsec, 20\arcsec$ at $z_l=0.1$). For discussion of modeling
lensing observables into the nonlinear regime, see \citet{Menard2003,
  Takada2003, Mandelbaum2006}. We will also assume these quantities
are constrained independently; \citet{Rozo2010} provide a treatment
of their covariance. Galaxies are assumed to lie at the centers of
their halos; see e.g., \citet{Johnston2007, George2012} for
approaches to treat miscentering.

Constraints on model profiles are derived in a Bayesian sense by
sampling the parameter space to maximize the posterior probability of
the data given the model and priors. The posterior for a set of model
parameters ${\bm \lambda}$ given the general data set ${\bm
  d}=\{\Sigma(R),\Delta\Sigma(R)\}$ is defined as 
\begin{equation}
\ln P({\bm \lambda}| {\bm d}) \propto
-\frac{1}{2}\chi^2_\mathcal{L}({\bm d}|{\bm \lambda}) + \ln{\mathcal P}({\bm \lambda}).
\end{equation}
The right-hand side consists of the prior ${\mathcal P}$ and the likelihood
\begin{equation}\begin{split}
\chi^2_{\mathcal L}({\bm d}|{\bm \lambda}) &=
\sum_{i}\left(\frac{\Sigma(R_i|{\bm \lambda}) -
      \Sigma(R_i|{\hat{\bm \lambda}})}{\sigma_{\Sigma}(R_i)}\right)^2 \\
&  + \left(\frac{\Delta\Sigma(R_i|{\bm \lambda}) -
      \Delta\Sigma(R_i|{\hat{\bm \lambda}})}{\sigma_{\Delta\Sigma}(R_i)}\right)^2
\end{split}\end{equation}
where $\hat{\bm \lambda}$ is the set of input model parameters. We
consider shear and magnification constraints separately in addition to
the combined data set, and remove data from the likelihood calculation
accordingly.

Sampling is performed using the
affine invariant Markov chain Monte Carlo (MCMC) algorithm of
\citet{Goodman2010} implemented in the {\tt emcee} code of
\citet{Foreman-Mackey2013}. We opt for a Monte Carlo approach rather
than Fisher information since it is more robust to non-Gaussianity and
nontrivial degeneracies in the likelihood surface, which are observed
in some cases. Priors are listed in Table~\ref{tab:model}. We use flat
priors with finite ranges to restrict the parameter space
searched, setting ${\mathcal P}$ to a constant within the specified
range and $-\infty$ outside. The ranges for $M_{\star}$ and $M_h$ are centered on the
fiducial values, while $\nu$ and $c$ are allowed to vary over a
plausible range borrowed from simulations. Since these parameters are
somewhat degenerate, we also test fixing the concentration to the
fiducial value, assuming that dark matter simulations can provide
reliable halo concentrations before any baryonic contraction or
expansion effects modify the halo.

Results in Figures~\ref{fig:contours_cfree}
and~\ref{fig:contours_cfixed} and Table~\ref{tab:constraints}. 

% **** FIG *****
\begin{figure*}[htb]
\epsscale{0.9}
\plotone{contours_alt4par_lsst_40_long.pdf}
\caption{[PLACEHOLDER: FIDUCIAL PARAMETERS AND PRIORS ARE NOT THE
  SAME AS TABLE~\ref{tab:model}] Posterior distributions for model
  parameters from MCMC fit with the fiducial Galaxy model, LSST survey
  parameters, $R_{\rm min}=40\,{\rm kpc}$, and $\Gamma=0.5(?)$. Contraints are from $\Sigma(R)$ (red),
  $\Delta\Sigma(R)$ (green), and the combined data (blue). The top
  column in each panel shows the one-dimensional posterior for each
  parameter while marginalizing over other parameters, and lower
  panels show the $68\%$ and $95\%$ contours for the joint posterior
  for each pair of parameters. $M_{\star}$ and $M_h$ are
  well-constrained (expect $M_{\star}$ with $\Sigma(R)$ only), while
  $c$ and $\nu$ are degenerate and unconstrained by the data.}
\label{fig:contours_cfree}
\end{figure*}
% **** FIG *****

% **** FIG *****
\begin{figure*}[htb]
\epsscale{0.8}
\plotone{contours_alt3par_lsst_40_long.pdf}
\caption{Same as Figure~\ref{fig:contours_cfree} but with $c$ fixed to
the fiducial value from simulations. If $c$ is known for the dark
matter halo prior to baryonic contraction or expansion, then $\nu$ can
be constrained.}
\label{fig:contours_cfixed}
\end{figure*}
% **** FIG *****

\begin{deluxetable*}{lccccccccc}
\tablecaption{Parameter Constraints}
\tablehead{\colhead{} & \multicolumn{4}{c}{Galaxies} & \colhead{} & \multicolumn{4}{c}{Clusters} \\
\colhead{Description} & \colhead{$\log(M_{\star})$} & \colhead{$\log(M_{h})$} & \colhead{$c$}
& \colhead{$\nu$} & \colhead{} & \colhead{$\log(M_{\star})$} & \colhead{$\log(M_{h})$} & \colhead{$c$} & \colhead{$\nu$}}
\startdata
\cutinhead{SDSS}
\sidehead{$R_{\rm min}=20\,{\rm kpc}$}
$c$ fixed, $\Gamma=0.5$ & 1 & 2 & 3 & 4 & & 1 & 2 & 3 & 4 \\
$c$ free, $\Gamma=0.5$ & 1 & 2 & 3 & 4 & & 1 & 2 & 3 & 4 \\
$c$ fixed, $\Gamma=1$ & 1 & 2 & 3 & 4 & & 1 & 2 & 3 & 4 \\
$c$ free, $\Gamma=1$ & 1 & 2 & 3 & 4 & & 1 & 2 & 3 & 4 \\
\sidehead{$R_{\rm min}=40\,{\rm kpc}$}
\cutinhead{LSST}
\sidehead{$R_{\rm min}=20\,{\rm kpc}$}
\sidehead{$R_{\rm min}=40\,{\rm kpc}$}
\enddata
\label{tab:constraints}
\end{deluxetable*}

\section{Discussion}

[PRELIMINARY DISCUSSION] It
looks like $M_{\star}$ and $M_h$ can be constrained well, but only one
of $c$ or $\nu$ can be constrained since they show a significant
degeneracy. I would argue that $c$ can be reasonably estimated from
dark matter simulations, so we can fix it to constrain $\nu$, which is
much less well-understood from hydrodynamical simulations. If the
\textit{wrong} $c$ is assumed, shear and magnification observables
give inconsistent constraints on $M_\star$ and $M_h$, which can
provide a useful test of the assumed $c$.  It is also interesting that
$M_\star$ can be constrained from lensing alone, which will help
determine the IMF. Even though we assume a Hernquist profile with a
fixed stellar mass to light ratio, the stellar component has a much
steeper profile than even an adiabatically contracted halo profile
(see Figure~\ref{fig:fiducial}), so moderate uncertainties in the
slope of the stellar component can be tolerated.


 %-------------- ACKNOWLEDGMENTS --------------------

\acknowledgments 

\mbox{~} % needed to prevent chopping off the last line above
%-------------- BIBLIO --------------------------------

%\pagebreak

%\bibliographystyle{yahapj}
\bibliographystyle{apj}
\bibliography{lensmodel}

 %----------------------------------------------------

\end{document}
