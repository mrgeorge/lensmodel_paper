%-------- DOCUMENT CLASS ------------
\documentclass[12pt]{emulateapj}

%-------- Packages ------------
%\usepackage[iop]{emulateapj}
\usepackage{amsmath,amssymb,natbib,graphicx}
\usepackage[usenames]{color}
\definecolor{Red}{rgb}{1.00,0.00,0.00}
\usepackage[colorlinks,urlcolor=blue,citecolor=black,linkcolor=blue]{hyperref} % for yahapj biblio links
%\pdfoutput=1                  % For the PDF latex compiler

% For US letter size
\special{papersize=8.5in,11in}
\setlength{\pdfpageheight}{\paperheight}
\setlength{\pdfpagewidth}{\paperwidth}


%-------- MACROS ------------

%\setlength{\topmargin}{0.7in} % uncomment when compiling on macbeth

%-------- TITLE COMMENT ------------

 %% You can insert a short comment on the title page using the command below.
%\slugcomment{ }
 
%-------- SHORT TITLE  ---------------------
 
%\shorttitle{}
%\shortauthors{}
 
%-------- BEGIN DOC  ---------------------
 
 \begin{document}
  
%-------- TITLE  ---------------------

 \title{On Measuring Inner Halo Profiles with Weak Lensing Shear and Magnification}

%-------- AUTHORS  ---------------------

 %% Use \author, \affil, and the \and command to format
 %% author and affiliation information.
 %% Note that \email has replaced the old \authoremail command
 %% from AASTeX v4.0. You can use \email to mark an email address
 %% anywhere in the paper, not just in the front matter.
 %% As in the title, use \\ to force line breaks.

\author{Matthew R. George}
\affil{Department of Astronomy, University of California, Berkeley, CA
  94720, USA}
\affil{Lawrence Berkeley National Laboratory, 1 Cyclotron Road,
  Berkeley, CA 94720, USA}
\email{mgeorge@astro.berkeley.edu}

%\submitted{Submitted to ApJ}

%-------- ABSTRACT  ---------------------
  
\begin{abstract}
\end{abstract}
 
%-------- KEY WORDS  ---------------------

%\keywords{}

%--------------------------------------------------------------
% INTRODUCTION
%--------------------------------------------------------------

\section{Introduction}

We want to measure the shape of the dark matter profile on small
scales. This will tell us how baryons have affected the dark matter
distribution, the assembly history of the baryons and dark matter as
well as feedback processes, and whether dark matter interacts to form
``cored'' profiles as opposed to the ``cuspy'' profiles seen in cold
dark matter simulations. Measurements of the mass distribution on
scales comparable to the effective radius of a galaxy can also
constrain the mass-to-light ratio of the stellar population which is
uncertain by a factor of $\sim2$ depending on the initial mass
function which is generally assumed.

Various observational probes have been used to constrain the mass
profile of galaxies and dark matter halos, typically with dynamical
measurements within a few kpc of the galaxy center, stong lensing arcs
a bit farther out, and weak lensing on larger scales to measure the
total mass of the halo \citet{Sand2004, Newman2013a, Auger, Bolton, Schulz2010}.
While central velocity dispersions or rotation curves can be measured
for large samples of galaxies, detailed kinematic measures on scales
$\gtrsim10$~kpc are difficult and strong lenses exist in only a sparse
sample of the galaxy population. Weak lensing, on the other hand, can
probe the average profile for large samples of galaxies, allowing
studies of population differences and redshift evolution. For
instance, the assembly histories of disk and elliptical galaxies may
differ and hydrodynamical simulations predict significant differences
in the inner profiles of their dark matter halos. 
The aim of this letter is to investigate how well weak lensing shear
and magnification can constrain the inner mass profile of dark matter
halos and galaxies.
 
Magnification and shear represent different components of the
lensing distortion matrix. [EQ] They are related to different aspects of
the surface mass density profile. [EQ]

Several authors have studied the complementarity of shear and
magnification, primarily for measuring halo masses
\citep{Bartelmann1996, Bridle1998, Schneider2000, vanWaerbeke2010b,
  Rozo2010, Umetsu2011} or probing the matter distribution on
cosmological scales \citep{vanWaerbeke2010a, Casaponsa2013,
  Duncan2013, Krause2013}. Combining shear and magnification increases
the statistical precision of a lensing experiment and also enables
tests of systematic effects which differ between probes.

Magnification can be measured using sizes, fluxes, or by combining the
two \citep{}. Magnification has recently been measured with
signal-to-noise approaching that from shear \citep{Hildebrandt2009,
  Menard2010, Huff2011, Ford2012, Schmidt2012}.



 %-------------- ACKNOWLEDGMENTS --------------------

\acknowledgments 

\mbox{~} % needed to prevent chopping off the last line above
%-------------- BIBLIO --------------------------------

%\pagebreak

%\bibliographystyle{yahapj}
\bibliographystyle{apj}
\bibliography{lensmodel}

 %----------------------------------------------------

\end{document}
